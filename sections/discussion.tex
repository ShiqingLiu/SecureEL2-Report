\section{Discussion}

we discussion the differences between S-EL2 and related works. List at least three novelties. 

% vTZ is a 
% vTZ的VM和vTEE都跑在NW, 但是每次改页表都得做世界切换, 在SEL2就不用, vTZ有多个VM和多个vTEE,VM之间的切换,vTEE之间的切换,VM和vTEE之间的切换,都要一次世界切换.
% 但在SEL2中,只有VM和vTEE之间的切换是需要世界切换的, vTZ需要修改NW hypervisor的binary.

% Please add the following required packages to your document preamble:
% \usepackage[table,xcdraw]{xcolor}
% If you use beamer only pass "xcolor=table" option, i.e. \documentclass[xcolor=table]{beamer}
% Please add the following required packages to your document preamble:
% \usepackage{booktabs}
\begin{table}[]
    \begin{tabular}{@{}lllllll@{}}
    \toprule
    \textbf{}                & \textbf{vTZ} & \textbf{TEEv} & \textbf{TDX} & \textbf{SEV} & \textbf{CCA} & \textbf{SecureEL2} \\ \midrule
    Architecture             & Arm          & Arm           & Intel        & AMD          & Arm          & Arm                \\
    modify code in TEE (EL0) & yes          & yes           &              & no           &              & no                 \\
    modify code in TEE (EL1) & no           & no            &              & no           &              & no                 \\
    modify normal world code & yes          & yes           &              &              &              & yes                \\ \bottomrule
    \end{tabular}
    \end{table}

% \begin{table*}[]
%     \begin{tabular}{llllllll}
%     \hline
%     \textbf{TEE} & \tabincell{c}{hardware\\ supported} & \tabincell{c}{modify TEE-kernel} & \tabincell{l}{support multiple\\ types of TEE-kernel} & \tabincell{c}{modify\\ hypervisor} & \tabincell{c}{transparent\\ to TA and REE} & \tabincell{c}{implemented\\ security hardware} & \tabincell{c}{architecture} \\ \hline
%     vTZ          & yes                         & yes                        & yes                                           & yes                        & yes                                & TrustZone                              & Arm                   \\
%     TEEv         & No                          & yes                        & yes                                           & yes                        & yes                                & TrustZone                              & Arm                   \\
%     Secure EL2   & yes                         & no                         & yes                                           & yes (a litttle)            & yes                                & TrustZone                              & Arm                   \\
%     TDX          & yes                         & -                          & -                                             & -                          &                                    &                                        &                       \\
%     SEV          &                             &                            &                                               &                            &                                    &                                        &                       \\
%     CCA          &                             &                            &                                               &                            &                                    &                                        &                       \\ \hline
%     \end{tabular}
%     \caption{Comparison with current TEE virtualization architectures.}
% 	\label{tab:comptee}
%     \end{table*}

Our project Secure EL2 has flowing three novelties compare to existing jobs, performance, compatibility, security. Seucre El2 has a better performance compared to vTZ and TEEv. There are two main reasons. vTZ will have a world swtiching as long as hypervisor wants to access page table, becuase SMM in Secure World controls the exclusively controls memory mapping and the sensitive instruction to access page table in Normal World hypervisor is replaced to the invocation to SMM. It is not that works in Secure EL2.  Secondly, the switch between VMs and VMs, vTEEs and vTEEs will also lead to a world switching, since the monitor, who is called SWS in vTZ and TEE-visor in TEEv, will verify the legality of switches. But Secure EL2 needn't do that. Becuase the switching between VMs and VMs is controlled by Normal World hypervisor, usually KVM, and the switching between vTEE and vTEE is controlled by SPM, usually Hafnium. Both are not related to world switching.

Another novelty is compatibility. A important advantages of Secure EL2 is that it make least impacts to the software running on Secure World. To meet the requirement of exclusively control the memory mapping, vTZ and TEEv make alterations to restrict code on EL1 and Secure EL1, manually or automatically. Although VMM in Secure EL2 may be modified to establish the communication to SPM, but it is much less than vTZ.

% Security. 