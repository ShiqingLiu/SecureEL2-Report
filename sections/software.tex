\section{Software Simulation Based TEE Virtualization(?)}

The software simulation base TEE virtualization is to create a virtual TEE and
simulate a part of, even all of TEE functionalities in a software way. The
important component of TEE, such as interrupt controller and memory controller,
is achieve by the software. The represensative is vTZ. 

\subsection{vTZ}

vTZ is proposed by Zhichao Hua in 2017 on usenix conference. TrustZone does not
provide any hardware support for virtualization at that time. All VMs in normal
world should shared the same TEE and it leads to a serious security issue for
breaking the isolation of VMs. vTZ aims to virtualize TrustZone and provide each
VM an independent TEE. Different from other jobs, it make use of the existed TEE
architecture instead of design a new one. vTZ seperate the funcutionality and
security by running another VM in normal world as guest TEE to serve the VM. The
security is ensure by the TrustZone hardware isolation. In details, a tiny
monitor in secure world control the memory mapping and world switching. A
Constrained Isolated Execution Environment (CIEE) provides isolation.

% vTZ design graphs
\begin{figure}
    \centering
    \includegraphics[scale=0.90]{vTZ_design.png}
    \caption{vTZ system designs}
    \label{fig:vTZ}
\end{figure}

The design of vTZ is shown in figure. There are four secure module in vTZ, Secure Memory Mapping (SMM), Secure World Switching (SWS), Control Flow Lock (CFLock) ,
and Constrained Isolated Execution Environment (CIEE). Running in secure world,
SMM controls all memory mappings. To achieve that, the hypervisor is not allow
to contain the sensitive instruction to access and control the translation
table, not to mention the first level translation and second level translation.
Instead, the sensitive instruction is replaced to the invocation to SMM. SMM
will check whether the request is legal. SWS is responsible for the world switching and it is another module in secure world. It will check the VID of the VMs and restore the correct VM. As long as SWS ensure the correction of the switch from VMs to the hypervisor and the hypervisor to VMs, all switches in vTZ will be covered, because switches between VMs are composed of both switches. As a module in normal world hypervisor, CFLock is used to prevent the execption control flow from tempering by protecting exception vector table. CFlock can used to ensure SWS will eventually handle all switches. The last module is CIEE, a region in normal world EL2, running with the hypervisor, serves the guest-TEE. CIEEs contiain the software implementations of TrustZone. In other words, vTZ does not utilize all the hardware in considerations of the support of the virtualization. To protect the isolation and security of CIEE, CIEE is designed to execute atomically and independently. The CIEE itself is also verifiable. As a result of it, CIEE can't be falsified.

Base on these four modules, vTZ reaches the goal of secure boot, memory isolation, and privilege isolation.

The advantages of vTZ is high efficiency and high security. In vTZ evaluation, the performance overhead of applications is low compared to the native environment, TrustZone, and Xen hypervisor. For security, vTZ has a TCB of thousands lines of code, which is far less than one who putting a entire hypervisor and guest TEE into secure world. And vTZ utilizes TrustZone to protect itself. In details, the privilege mode and secure world of TrustZone make the isolation and security of CIEE, SWS and SMM. In a result, vTZ successfully defenses several kinds of attacks, such as code-reuse attacks, DMA attacks, debugging attacks and code tempering attacks.

However, vTZ also has some disadvantages. vTZ is not compatible to the exisiting commercial hypervisor. In vTZ, the hypervisor is not allow to have sensitive instructions related to address translations. In addition, the module, CFLock should run in hypervisor to support vTZ. So the hypervisor must modified their source code in order to run on vTZ. That is a huge disadvantage to the promotion and application.

In conclusion, vTZ is mostly close to our goals Secure EL2 project. vTZ is also a close competiter of our project. We want to compare the performance with vTZ and proof the advantage of our one.

% vTZ 的background 和discussion is valuable 

% temp{vTZ advantages and disadvantages, our design section: }

\subsection{TEEv}

TEEv \cite{li2019teev} is another TEE virtualization architecture to provide restricted TEEs, i.e. vTEEs. These vTEEs are concurrent in runtime and isolated from each other. The vTEEs is not like traditional TEEs who can access the REE's resources. They can only access their own assest(More about vTEE). A tiny hypervisor, TEE-visor, running in Secure World to manage these vTEEs and to provide isolation between vTEEs. TEE-visor allows each vTEE from different manufacturers to host their own TA. 

To address the challenge of the lack of TrustZone virtualization support, TEEv deploys the TEE-visor and vTEEs at the same privilege level, secure EL1. In addition, TEEv modified the commercial TEE by scanning vTEE's binary to remove the MMU-related instructions, so that TEE-visor can exclusively control the MMU. Controling the MMU, TEE-visor can protect itself in security and confidentiality, and isolate vTEEs by protecting each vTEE's entries. As a result of removing the sensitive instruction, TEE-visor is responsible for the communication between vTEE and corresponding client application (CA). In details, TEE-visor handles the request from CAs and verify the page access from TA to CA. For isolation of vTEEs and REEs, TEE-visor manage the page table exclusively in order that each vTEE and each REE can only access their own page. Similar to vTZ, TEE-visor protect the kernel of vTEEs by mapping their kernel's page as read only. (other details)

The advantage of TEEv is high security and high flexibility. The vTEE in TEEv is not a traditional TEE and it is restricted. If a vTEE is malicious, it can not affect the other vTEEs and REEs. Its TCB is also low. The TEE-visor only contians 3800 lines of C and assembly code. Each TEE should only trust their own code base and TEE-visor. (CVEs and attacks) For the flexibility, TEEv allows to install multiple vTEEs and support each own TA. ()

% The disadvantage of TEEv is 


